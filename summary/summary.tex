



\documentclass[11pt]{article}
% Default margins are too wide all the way around. I reset them here
\setlength{\topmargin}{-.5in}
\setlength{\textheight}{9in}
\setlength{\oddsidemargin}{.125in}
\setlength{\textwidth}{6.25in}

% Nicer paragraph style
\setlength{\parskip}{\baselineskip}%
\setlength{\parindent}{0pt}%

% Font
%\usepackage[urw-garamond]{mathdesign}
%\usepackage[T1]{fontenc}

\usepackage[usenames,dvipsnames,svgnames,table]{xcolor}
\usepackage{amsmath}
\usepackage{amssymb}
\usepackage{graphicx}
\usepackage{hyperref}
\usepackage{lscape}

% Config
\graphicspath{{./images/}}
\hypersetup{
    colorlinks=true,
    bookmarks=true,
    unicode=true,
    linkcolor=black,
    citecolor=black,
    filecolor=black,
    urlcolor=black
}

\begin{document}



\title{A Comparison Comparison:\\
\small{for comparing how to compare comparable things}}
\author{Stephen Wattam}
\date{\today}
\maketitle

% Table of contents
\tableofcontents{}
\pagebreak




% % =============================================================================
% %                               <template>
% % =============================================================================
% \section{Template} %name of group of statistics
% \subsection{Description}
% % Describe the rationale behind the measure:
% %  1) Who developed it, when?
% %  2) Why? What field was it originally developed for?
% %  3) To solve what problem?
% 
% \subsection{Properties}
% \begin{tabular}{| l || l |}
%     \hline
%     {\bf Property} & {\bf Description} \\
%     \hline
%     % What values may the statistic take
%     Limits & [-1, 1] \\ \hline
% 
%     % Is this method only valid under certain
%     % distributional assumptions?
%     Distributional Assumptions& 2 \\ \hline
% 
%     % What types of data may this run on?
%     Data type & nominal, ordinal, interval, ratio \\ \hline
% 
%     % Is the method suitable for small frequencies?
%     Low Frequency & 8 \\ \hline
% 
%     % Is the method suitable for nonsquare comparisons?
%     Squareness Assumptions & 8 \\ \hline
%     
%     % Is the method symmetric?
%     Symmetry? & Yes \\ \hline
% 
%     % Finally, any other assumptions
%     % x & x \\ \hline
% \end{tabular}
% 
% 
% \subsection{Formula}
% % How do we compute the statistic?
% 
% \subsection{Intrepretation}
% % How to intrepret the output: what does it mean?
% %  1) What range of values does it output?
% %  2) In what circumstances does/can it hit extremis?
% %  3) What does the number mean re. the data?
% 
% \subsection{Critical Values}
% % Is it possible to find critical values for this measure?
% %  1) If so, how? (brief mention is adequate for now)
% %  2) In what contexts are the values valid?
% % =============================================================================

\appendix




\section{Table of Symbols}
\begin{table}[h]
\centering
\begin{tabular}{|l|l|}
\hline
\textbf{Symbol} & \textbf{Comparison Method} \\ \hline
$\rho$              & Pearson's rho \\ \hline
$\rho_S$            & Spearman's rho \\ \hline
$\tau_a$            & Kendall's Tau-a \\ \hline
$\tau_b$            & Kendall's Tau-b \\ \hline
$\tau_c$            & Stuart's Tau-c \\ \hline
$\gamma$, G         & Goodman \& Kruskal's gamma\\ \hline
$\kappa$            & Cohen's kappa \\ \hline
$\kappa_F$          & Fleiss' kappa \\ \hline
$\lambda$           & Goodman \& Kruskal's lambda\\ \hline
$U$                 & Theil's U/Uncertainty coefficient\\ \hline
$\tau_{GK}$         & Goodman \& Kruskal's tau\\ \hline
$D_{KL}$            & Kullback-Leibler divergence\\ \hline
$D$                 & Somers' D\\ \hline
$\eta$              & Eta \\ \hline
$\chi^2$            & Pearson's chi-squared\\ \hline
$\phi$              & Matthews correlation coefficient\\ \hline
$V$                 & Cram\'er's V\\ \hline
$CC$                & Contingency coefficient \\ \hline
$OR$                & Odds ratio\\ \hline
$\alpha$            & Cronbach's alpha\\ \hline
$T$                 & Tschuprow's T\\ \hline
$F_\beta$           & F-score (where $\beta$ is the weighting parameter)\\ \hline
$U_{MW}$            & Mann-Whitney-Wilcoxon U\\ \hline
$D_{KS}$            & Kolmogorov-Smirnov D\\ \hline
$\Lambda$           & Likelihood ratio\\ \hline
$cos$               & Cosine Similarity\\ \hline
$D_\%$              & Percentage difference \\ \hline
$D_E$               & Euclidean distance\\ \hline
$D_M$               & Manhattan (taxicab) distance\\ \hline
$T_v$               & Tversky index\\ \hline
$p$                 & Fisher's exact test score (simple probabilty)\\ \hline
$K_{\alpha}$        & Kilgarriff's simple maths (where $\alpha$ is the saturation parameter)\\ \hline
%$ $                & \\ \hline
\end{tabular}
\caption{Symbols for each method}
\label{tab:symbols}
\end{table}



\section{Comparison}
\begin{landscape}
\begin{table}[h]
\centering
\begin{tabular}{|l|l|l|l|l|l|l|l|l|}
\hline
\textbf{Symbol} & \textbf{Name} & \textbf{Rationale} & \textbf{Limits} & \textbf{Squareness?} & \textbf{Sensitivity} & \textbf{Low-Frequency?} & \textbf{Symmetric?} & \textbf{Formula} \\
\hline

% TODO

\end{tabular}
\caption{Symbols for each method}
\label{tab:symbols}
\end{table}

\end{landscape}



% 'The needless assumption of normality in Pearson's r' http://psycnet.apa.org/journals/amp/12/10/623/


% =============================================================================
% =============================================================================
% =============================================================================

% =============================================================================
\bibliography{comparison}
\bibliographystyle{plain}
\end{document}
